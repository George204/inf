\documentclass{article}
\usepackage{polski}
\usepackage[utf8]{inputenc}
\usepackage{tikz}
\usepackage{listings}
\usepackage[a4paper, total={18cm, 28cm}]{geometry}
\usetikzlibrary{shapes.geometric, arrows}
\usepackage{tcolorbox}
\usepackage{fontspec}
\setmainfont{JetBrains Mono semibold}
\pagestyle{empty}

\tikzstyle{startstop} = [ellipse, minimum width=1.5cm, minimum height=0.5cm,text centered, draw=black, fill=red!30]
\tikzstyle{trap} = [trapezium, trapezium left angle=70, trapezium right angle=110, minimum width=1.0cm, minimum height=0.5cm, text centered, draw=black, fill=blue!30]
\tikzstyle{pro} = [rectangle, minimum width=1.5cm, minimum height=0.5cm, text centered, draw=black, fill=orange!30]
\tikzstyle{romb} = [diamond, minimum width=1.5cm, minimum height=0.5cm, text centered, draw=black, fill=green!30]
\tikzstyle{arr} = [thick,->,>=stealth]

\newtcolorbox{temp}[2][]{colbacktitle=white,
colback=white,coltitle=black,
title={#2},#1}



\begin{document}

\section*{zadanie 4.1}

\begin{temp}{\large cpp}  
\begin{verbatim}
#define max 101
int main(){
    srand(time(NULL));
    bool re = true;
    do{
        int size = input("Ile elementow chcesz podac:");        
        int tab[max];
        czytajt(tab, size);
        cout<<"Twoje liczby odwrotnie to: ";
        for(int i=0;i<size;i++){
            cout<<tab[size-i-1]<<" ";
        }
        jeszcze(re);
    }while(re==true);
    return 0;
}

\end{verbatim}
\end{temp}

\begin{temp}{\large python}  
\begin{verbatim}
from funk import *
max = 5

def main():
    re = True
    while re == True:
        size = inputh("Ile elemntuw chcesz podac:")
        tab = [0,]*size
        czytajt(tab,size)
        piszt(tab,size)
        print(f"Twoje liczby odwrotnie to :",end="")
        for i in reversed(tab):
            print(f"{i},", end="")
        print()
        re = jeszcze()

if __name__ == "__main__":
    main()


\end{verbatim}
\end{temp}

\section*{zadanie 4.2}

\begin{temp}{\large cpp}  
\begin{verbatim}
#define max 100

string sprczyw(int n,int tab[max] ,int size){
    for(int i=0;i<size;i++){
        if( tab[i] == n){return "tak";}
    }
    return "nie";

}
int main(){
    srand(time(NULL));
    bool re = true;
    do{
        int size = input("Jaki ma byc rozmiar tablicy: ");        
        int tab[size];
        cout<<"podajemy ? t or n\n";
        char ans = getin();
        if(ans == 't'){
            czytajt(tab, size);
        }
        else{
            lost(tab, size);
        }
        piszt(tab, size);
        int szuk = input("jako liczbe chcesz sprawdzic?: ");
        cout<<"czy jest:"<<sprczyw(szuk,tab,size);
        jeszcze(re);
    }while(re==true);
    return 0;
}

\end{verbatim}
\end{temp}

\begin{temp}{\large python}  
\begin{verbatim}
from funk import *
max = 5

def sprczyw(n, tab, size):
    for i in range(size):
        if tab[i] == n:
            return "tak"
    return "nie"

def main():
    re = True
    while re == True:
        size = inputh("Jaki ma być rozmiar tablicy:")
        tab = [0,]*size
        print("podajemy ? (T/N)")
        ans = getin()
        if ans == b't':
            czytajt(tab,size)
        else:
            lost(tab,size)
        piszt(tab,size)
        szuk = inputh("jako liczbe chcesz sprawdzic?:")
        print(f"czy jest:{sprczyw(szuk,tab,size)}")
        re = jeszcze()

if __name__ == "__main__":
    main()


\end{verbatim}
\end{temp}

\section*{zadanie 4.3}

\begin{temp}{\large cpp}  
\begin{verbatim}
#define max 100


string sprczyw(int n,int tab[max] ,int size){
    for(int i=0;i<size;i++){
        if( tab[i] == n){return "tak index="+to_string(i);}
    }
    return "nie";

}
int main(){
    srand(time(NULL));
    bool re = true;
    do{
        int size = input("Jaki ma byc rozmiar tablicy: ");        
        int tab[size];
        cout<<"podajemy ? t or n\n";
        char ans = getin();
        if(ans == 't'){
            czytajt(tab, size);
        }
        else{
            lost(tab, size);
        }
        piszt(tab, size);
        int szuk = input("\njako liczbe chcesz sprawdzic?: ");
        cout<<"czy jest:"<<sprczyw(szuk,tab,size);
        jeszcze(re);
    }while(re==true);
    return 0;
}

\end{verbatim}
\end{temp}

\begin{temp}{\large python}  
\begin{verbatim}
from funk import *
max = 5

def sprczyw(n, tab, size):
    for i in range(size):
        if tab[i] == n:
            return f"tak index={i}"
    return "nie"

def main():
    re = True
    while re == True:
        size = inputh("Jaki ma być rozmiar tablicy:")
        tab = [0,]*size
        print("podajemy ? (T/N)")
        ans = getin()
        if ans == b't':
            czytajt(tab,size)
        else:
            lost(tab,size)
        piszt(tab,size)
        szuk = inputh("jako liczbe chcesz sprawdzic?:")
        print(f"czy jest:{sprczyw(szuk,tab,size)}")
        re = jeszcze()

if __name__ == "__main__":
    main()


\end{verbatim}
\end{temp}

\section*{zadanie 4.4}

\begin{temp}{\large cpp}  
\begin{verbatim}
#define max 100


string sprczyw(int n,int tab[max] ,int size){
    string odp = "Nie";
    for(int i=0;i<size;i++){
        if( tab[i] == n){
            if(odp == "Nie"){odp = "Tak indexy:";}
            odp = odp+to_string(i)+",";
        }
    }
    return odp;
}
int main(){
    srand(time(NULL));
    bool re = true;
    do{
        int size = input("Jaki ma byc rozmiar tablicy: ");        
        int tab[size];
        cout<<"podajemy ? t or n\n";
        char ans = getin();
        if(ans == 't'){
            czytajt(tab, size);
        }
        else{
            lost(tab, size);
        }
        piszt(tab, size);
        int szuk = input("\njako liczbe chcesz sprawdzic?: ");
        cout<<"czy jest:"<<sprczyw(szuk,tab,size);
        jeszcze(re);
    }while(re==true);
    return 0;
}

\end{verbatim}
\end{temp}

\begin{temp}{\large python}  
\begin{verbatim}
from funk import *
max = 5

def sprczyw(n, tab, size):
    odp = "Nie"
    for i in range(size):
        if tab[i] == n:
            if odp == "Nie":
                odp = "Tak indexy:"
            odp = odp+f"{i},"
    return odp

def main():
    re = True
    while re == True:
        size = inputh("Jaki ma być rozmiar tablicy:")
        tab = [0,]*size
        print("podajemy ? (T/N)")
        ans = getin()
        if ans == b't':
            czytajt(tab,size)
        else:
            lost(tab,size)
        piszt(tab,size)
        szuk = inputh("jako liczbe chcesz sprawdzic?:")
        print(f"czy jest:{sprczyw(szuk,tab,size)}")
        re = jeszcze()

if __name__ == "__main__":
    main()


\end{verbatim}
\end{temp}

\section*{zadanie 4.5}

\begin{temp}{\large cpp}  
\begin{verbatim}
#define max 100


string sprczyw(int tab[max] ,int size){
    for(int i=0;i<size;i++){
        for(int j = i+1;j<size;j++){
            if( tab[i] == tab[j] ){return "Tak";}
        }
    }
    return "Nie";
}
int main(){
    srand(time(NULL));
    bool re = true;
    do{
        int size = input("Jaki ma byc rozmiar tablicy: ");        
        int tab[size];
        cout<<"podajemy ? t or n\n";
        char ans = getin();
        if(ans == 't'){
            czytajt(tab, size);
        }
        else{
            lost(tab, size);
        }
        piszt(tab, size);
        cout<<"\nCzy sie powtarza: "<<sprczyw(tab,size);
        jeszcze(re);
    }while(re==true);
    return 0;
}

\end{verbatim}
\end{temp}

\begin{temp}{\large python}  
\begin{verbatim}
from funk import *
max = 5

def sprczyw(tab, size):
    conv = set(tab)
    if len(conv) < size:
        return "Tak"
    return "Nie"

def main():
    re = True
    while re == True:
        size = inputh("Jaki ma być rozmiar tablicy:")
        tab = [0,]*size
        print("podajemy ? (T/N)")
        ans = getin()
        if ans == b't':
            czytajt(tab,size)
        else:
            lost(tab,size)
        piszt(tab,size)
        print(f"czy sie powetarza:{sprczyw(tab,size)}")
        re = jeszcze()

if __name__ == "__main__":
    main()


\end{verbatim}
\end{temp}

\section*{zadanie 4.6}

\begin{temp}{\large cpp}  
\begin{verbatim}
#define max 100


void znajdz(int tab[max][max], int n, int m){
    int maxim[3] = {0,0,tab[0][0]};
    int minim[3] = {0,0,tab[0][0]};
    for(int i=0;i<n;i++){
        for(int j=0;j<m;j++){
            int wart = tab[i][j]; 
            if(wart<minim[2]){
                minim[0] = i;
                minim[1] = j;
                minim[2] = wart;
            }
            if(wart>maxim[2]){
                maxim[0] = i;
                maxim[1] = j;
                maxim[2] = wart;
            }
        }
    }
    cout<<"wartosc maksymalna:"<<maxim[2]<<" index: "<<maxim[0]<<","<<maxim[1]<<endl;
    cout<<"wartosc minimalna:"<<minim[2]<<" index: "<<minim[0]<<","<<minim[1]<<endl;
}
int main(){
    srand(time(NULL));
    bool re = true;
    do{
        int tab[max][max];
        int kol = input("Ile kolumn: ");
        int wier = input("Ile wierszy: ");
        cout<<"podajemy ? t or n\n";
        char ans = getin();
        if(ans == 't'){
            czytajt2(tab,wier, kol);
        }
        else{
            lost2(tab, wier, kol);
        }
        piszt2(tab, wier, kol);
        znajdz(tab, wier, kol);
        jeszcze(re);
    }while(re==true);
    return 0;
}

\end{verbatim}
\end{temp}

\begin{temp}{\large python}  
\begin{verbatim}
from funk import *
max = 5

def znajdz(tab, n, m):
    maxim = (0,0,tab[0][0])
    minim = (0,0,tab[0][0])
    for i in range(n):
        for j in range(m):
            wart = tab[i][j]
            if wart < minim[2]:
                minim = (i,j,wart)
            if wart > maxim[2]:
                maxim = (i,j,wart)
    print(f"wartosc maksymalna:{maxim[2]} index: ({maxim[0]},{maxim[1]})")
    print(f"wartosc minimalna:{minim[2]} index: ({minim[0]},{minim[1]})")

def main():
    re = True
    while re == True:
        kol  = inputh("Ile kolumn: ")
        wier = inputh("Ile wierszy: ")
        tab = [[0,]*kol]*wier
        print("podajemy ? (T/N)")
        ans = getin()
        if ans == b't':
            czytajt2(tab,wier,kol)
        else:
            lost2(tab,wier,kol)
        piszt2(tab,wier,kol)
        znajdz(tab,wier,kol)    
        re = jeszcze()

if __name__ == "__main__":
    main()


\end{verbatim}
\end{temp}

\section*{zadanie 4.7}

\begin{temp}{\large cpp}  
\begin{verbatim}
#define max 100


bool pierw(int n){
    bool out = true;
    for(int i=2;i<n/2+1;i++){
        if(n%i==0){out = false;break;}
    }
    return out;
}
void znajdz(int tab[max][max], int n, int m){
    int suma = 0;
    float il = 0;
    for(int i=0;i<n;i++){
        for(int j=0;j<m;j++){
            int wart = tab[i][j];    
            if(pierw(wart)){
                cout<<"("<<i<<","<<j<<") "<<wart<<" jest pierwsza\n";
                suma = suma + wart;
                il++;
            }
        }
    }
    cout<<fixed<<setprecision(2)<<"suma: "<<suma<<"\nsrednia: "<<suma/il;
}
int main(){
    srand(time(NULL));
    bool re = true;
    do{
        int tab[max][max];
        int kol = input("Ile kolumn: ");
        int wier = input("Ile wierszy: ");
        cout<<"podajemy ? t or n\n";
        char ans = getin();
        if(ans == 't'){
            czytajt2(tab,wier, kol);
        }
        else{
            lost2(tab, wier, kol);
        }
        piszt2(tab, wier, kol);
        znajdz(tab, wier, kol);
        jeszcze(re);
    }while(re==true);
    return 0;
}

\end{verbatim}
\end{temp}

\begin{temp}{\large python}  
\begin{verbatim}
from funk import *
max = 5

def pierw(n):
    for i in range(2,int(n/2+1)):
        if n%i == 0:
            return False
    return True

def znajdz(tab, n, m):
    suma = 0
    il = 0
    for i in range(n):
        for j in range(m):
            wart = tab[i][j]
            if pierw(wart):
                print(f"({i},{j}) {wart} jest pierwsza")
                suma += wart
                il += 1
    print(f"suma: {suma}\nsrednia: {suma/il:.2f}")

def main():
    re = True
    while re == True:
        kol  = inputh("Ile kolumn: ")
        wier = inputh("Ile wierszy: ")
        tab = [[0,]*kol]*wier
        print("podajemy ? (T/N)")
        ans = getin()
        if ans == b't':
            czytajt2(tab,wier,kol)
        else:
            lost2(tab,wier,kol)
        piszt2(tab,wier,kol)
        znajdz(tab,wier,kol)    
        re = jeszcze()

if __name__ == "__main__":
    main()


\end{verbatim}
\end{temp}

\section*{zadanie 4.8}

\begin{temp}{\large cpp}  
\begin{verbatim}
#define max 100



void sortt(int tab[max], int size){
    int zmiana = 0;
    do{
        zmiana = 0;
        for(int i=0;i<size-1;i++){
            if(tab[i]>tab[i+1]){
                zmiana = 1;
                int schowek = tab[i];
                tab[i]=tab[i+1];
                tab[i+1]=schowek;
            }
        }
    }while (zmiana == 1);
}

int main(){
    srand(time(NULL));
    bool re = true;
    do{
        int size = input("Jaki ma byc rozmiar tablicy: ");        
        int tab[size];
        cout<<"podajemy ? t or n\n";
        char ans = getin();
        if(ans == 't'){
            czytajt(tab, size);
        }
        else{
            lost(tab, size);
        }
        piszt(tab, size);
        sortt(tab, size);
        piszt(tab, size);
        jeszcze(re);
    }while(re==true);
    return 0;
}

\end{verbatim}
\end{temp}

\begin{temp}{\large python}  
\begin{verbatim}
from funk import *
max = 5

def sortt(tab):
    return sorted(tab)

def main():
    re = True
    while re == True:
        size = inputh("Jaki ma być rozmiar tablicy:")
        tab = [0,]*size
        print("podajemy ? (T/N)")
        ans = getin()
        if ans == b't':
            czytajt(tab,size)
        else:
            lost(tab,size)
        piszt(tab,size)
        tab = sortt(tab)
        piszt(tab,size)
        re = jeszcze()

if __name__ == "__main__":
    main()

\end{verbatim}
\end{temp}
\end{document}
